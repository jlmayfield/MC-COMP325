\documentclass[10pt]{article}
\usepackage{amsmath}
\usepackage{setspace}


\setlength{\textheight}{9in} \setlength{\topmargin}{-.5in}
\setlength{\textwidth}{6.5in} \setlength{\oddsidemargin}{0in}
\setlength{\evensidemargin}{0in}

\title{COMP 325 - Language Research Project}
\author{  }
\date{Fall 2015}


\begin{document}
\maketitle

You'll be carrying out some research on a language of your choosing.  Towards the end of the semester you'll be presenting your language and your findings to the class.

\section{The Research}

For your language of choice you must be prepared to present and discuss the following topics:
\begin{itemize}
\item History and Background
\item Language Syntax and Semantics 
\item Language Pragmatics
\end{itemize}



\subsection{History and Background}

A good chunk of your research should address the history and background of your chosen language.  This includes:
\begin{itemize}
\item Brief History of the Language
\item Who Developed the Language and Why
\item Who used/uses the language and Why
\item What languages are related to it and how
\item Any well known projects/programs that were implemented using the language
\end{itemize}
This is the non-technical portion of the report.  The main goal of this section is for you to paint the picture of why this language exists and why, as students of programming and computing, we should be interested in it.  

\subsection{Language Syntax and Semantics}

This section should delve into the nuts and bolts of the language.  More specifically its syntax, semantics, and pragmatics. Things to look at include (but are not limited to):
\begin{itemize}
\item Is it declarative, imperative, or somewhere in between?
\item What concepts and paradigms does it support? For example:
\begin{itemize}
\item State and Named State
\item Records/Indexed Data Structures
\item Objects
\item Lexically Scoped Closures
\item Laziness/Lazy Evaluation 
\item Types and Variable Typing
\item Recursion and Tail-Call Optimization
\item Garbage Collection
\end{itemize}
\end{itemize}
To help facilitate this analysis, provide code examples whenever possible.  Be certain to cite any examples that you pulled from one of your sources.  

\subsection{Language Pragmatics}

This research covers current or past best-practices for development with your language.  Look at things like,
\begin{itemize}
\item Tools available for the language (Compilers, Interpreters, IDEs, etc.)
\item Documented style-guides and language idioms
\end{itemize}
Your goal is to establish how one effectively and efficiently utilizes, or utilized, your language in practice.  


\section{Logistics}

\begin{enumerate}
\item \textbf{Due 10/9} \textit{Annotated Bibliography }
Your bibliography should include official documentation, general sources on the language's history and background, as well as information about notable projects.  It may also include third-party references and source code.  \textit{You will turn in a printed copy of your bibliography and give a five minute presentation of your sources discussing their credibility when appropriate. This counts as a homework grade}
\item \textbf{Due 11/13} \textit{``Quicksort'' Paper}
Your paper should broadly cover the syntax, semantics, and pragmatics of your chosen language as discussed above. To keep things from being too general and vague, the focal point of your paper should be the technical presentation of Quicksort, or some suitably complex algorithm, in your language. This is not a paper only about quicksort in your language. This paper should be about the language and Quicksort is your concrete, example from which you draw out key points about syntax, semantics, and pragmatics. Papers must not exceed 5 pages. \textit{Submit a printed copy of the paper to the instructor and email a digital copy to the class. You will be peer-reviewing your classmates papers for homework.}
\item \textbf{Due 11/23-24} \textit{Final Presentations}
You will make a 15 minute, in-class presentation about your research.  This time should include a few minutes for questions.  
\end{enumerate}

\end{document}