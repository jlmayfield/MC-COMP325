\documentclass[10pt]{article}
\usepackage{amsmath}
\usepackage{setspace}
\usepackage{hyperref}

\setlength{\textheight}{9in} \setlength{\topmargin}{-.5in}
\setlength{\textwidth}{6.5in} \setlength{\oddsidemargin}{0in}
\setlength{\evensidemargin}{0in}

\title{Syllabus - COMP 325 - Organization of Programming Languages}
\author{  }
\date{Fall 2015}

\begin{document}
\maketitle

\section{Logistics}
\begin{itemize}
\item \textbf{Where: } Center for Science and Business, Room 303
\item \textbf{When: } MTWF,  1-1:50pm
\item \textbf{Instructor :} James \textit{Logan} Mayfield
\begin{itemize}
\item \textit{Office: } Center for Science and Business, Room 344
\item \textit{Phone: } 309-457-2200
\item \textit{Email: } lmayfield at MONMOUTHCOLLEGE dot EDU
\item \textit{Office Hours: } By Appointment
\end{itemize}
\item \textbf{Website: } \url{https://jlmayfield.github.io/MC-COMP325}
\item \textbf{Credits: } 1 course credit
\end{itemize}
\emph{Note: Parts of this Syllabus are subject to change based on specific class needs.}

\section{Texts}
%insert text #1
Krishnamurthi, Shriram. \textit{Programming and Programming Languages}. 2014. 
\begin{itemize}
\item Book (HTML): \url{http://papl.cs.brown.edu/2014/}
\end{itemize}

\vspace{.2in} 

%insert text #2 
\noindent
Krishnamurthi, Shriram. \textit{Programming Languages Application and Interpretation}. Second Edition. Creative Commons. 2012. 
\begin{itemize}
\item Main Site: \url{http://www.cs.brown.edu/~sk/Publications/Books/ProgLangs/2007-04-26/}
\item Book (HTML): \url{http://cs.brown.edu/courses/cs173/2012/book/}
\end{itemize}


\section{Programming Language and Environment}

We'll be diving in with the \textit{Pyret} language that accompanies PAPL. Pyret has an in-browser development environment and so may be used on any computer with a (modern) web-browser.
\begin{itemize}
\item \url{http://www.pyret.org/}
\item Style Guide: \url{http://cs.brown.edu/courses/cs173/2013/Pyret_Style_Guide.html}
\end{itemize}

\noindent
We might, from time to time, look at some \textit{Racket} as its used in PLAI.  If you need or want to update your DrRacket installation, here's the link.
\begin{itemize}
\item \url{http://www.racket-lang.org}
\item Style Guide: \url{http://cs.brown.edu/courses/cs017/content/docs/racket-style.pdf}
\end{itemize}

\section{Description and Content}
This course is an exploration of modern programming languages through the study and implementation of  interpreters for fundamental language features.  By implementing small languages with common PL features, students expand their skill set with both practical and theoretical knowledge.  To understand programming languages is to understand programming and computation as a whole.  A programming language is how we describe a computational process and study of the languages themselves helps to shed light on the inner workings of a computation.

\subsection{Content}

This course will, for the most part, follow the text.  Topics to be covered will include:
\begin{itemize}
\item Principles of Language Design and Implementation 
\item Arithmetic Expressions
\item Conditionals
\item Functions and Procedures
\item Records
\item State and Mutation
\item Garbage Collection
\item Types
\item Objects
\item Parametrized Types 
\item Type Inference
\end{itemize}

\section{Expectations and Policies}

You are expected to carry yourself in a mature and professional manner in this course. Towards this end, there are a few classroom policies by which you are expected to abide.
\begin{itemize}

\item \textit{Late Assignments: } In general, late assignments will \textit{not} be accepted.  If you feel you have a justified reason for the assignment being late you may set up an appointment to meet with the instructor and plead your case.  Situations beyond your control are understandable and exceptions can and will be made. 

\item \textit{Attendance: } \textbf{Repeated absences and late arrivals to class will quickly reduce your participation grade to zero.}  The occasional late arrival or missed class is one thing, but being habitually late and regularly missing classes is disruptive and not fair to your classmates.  

\item \textit{Participation: }  Cellphone and computer usage in class for non-class related activities is strongly discouraged.  All devices should be set to silent when in class.  If your usage of technology becomes a distraction to your classmates or your instructor, then your participation grade will suffer.  If you're not sure if your being a distraction, then err on the side of caution and assume your distracting someone.  Put another way, if the instructor or a classmate has to tell you you're distracting them, then you've already gone too far. 

\item \textit{Quality of Work:} There are several minimal requirements that your assignments must meet.
\begin{itemize}
\item \textit{Electronic Submissions}  Most of your work will be handed in electronically.  It is your responsibility to know and understand the system for doing so and to be sure your work has properly submitted. Not following the instructions for assignment submission can mean your assignment does not get submitted and will be considered late. 

\item \textit{Staples - } Assignments that take up more than one page must be stapled.  Unstapled assignments will either be returned to you to be stabled ASAP or points will be deducted.  

\item \textit{Neatness - }  Make every attempt to make your work neat and orderly:  label problems, avoid excessive scratching out of mistakes (use pencil if you are prone to errors) and if you use spiral bound paper tear off the edges. Put your name on your work!

\item \textit{Show Work - } Rarely are answers alone sufficient for full credit.  Show your work whenever prudent.  If you're unsure if work is needed, \textit{ask!}
\end{itemize}

\end{itemize}


\subsection{Collaboration}

In general, you are encouraged to make use of the resources available to you.  This means it is OK to seek help from a friend, tutor, instructor, internet, etc.  However, \textit{copying of answers and any act worthy of the label of ``cheating'' is never permissible!}  It is understandable that when you work with a partner or a group that the resultant product is often extremely similar.  This is acceptable but be prepared to be asked to defend your collaborations to the instructor.  \textit{You should always be able to reproduce an answer on your own, and if you cannot you likely \textbf{do not really known the material.}} All of the Monmouth College rules on academic dishonesty apply.  If you violate the rules be prepared to face the consequences of your actions.  

\section{Grades}

This courses uses a standard grading scale.  Assignments and final grades will not be curved except in rare cases when its deemed necessary by the instruction.  Percentage grades translate to letter grades as follows:

\begin{center}
\begin{small}
\begin{tabular}{lcl}
Score & & Grade \\ \hline
94-100 & & A \\
90-93 & & A- \\
88-89 & & B+ \\
82-87 & & B \\
80-81 & & B- \\
78-79 & & C+ \\
72-77 & & C \\
70-71 & & C- \\
68-69 & & D+ \\
62-67 & & D \\
60-61 & & D- \\
0-59 & & F 
\end{tabular}
\end{small}
\end{center}

You are always welcome to challenge a grade that you feel is unfair or calculated incorrectly.  Mistakes made in your favor will never be corrected to lower your grade.  Mistakes made not in your favor will be corrected.  \textit{Basically, after the initial grading your score can only go up as the result of a challenge.}

\subsection{Workload}
% number of/details on midterms, finals, project, homeworks, quizes, etc

\begin{itemize}
\item 2-3 Homework Assignments 
\item 5-7 Interpreters
\item 1 Paper with Presentation
\item 1 Final
\item 1 Midterm
\item 3 Quizzes
\end{itemize}

\subsection{Grade Weights}

Your final grade is based on a weighted average of particular assignment categories.  You should be able to estimate your current grade based on your scores and these weights.  You may always visit the instructor \textit{outside of class time} to discuss your current standing.  
\begin{itemize}
\item Homework 5\%
\item Interpreters 30\%
\item Paper + Presentation 20\%
\item Final 15\%
\item Midterm 10\%
\item Quizzes 10\%
\item Participation 10\%
% add more if needed
\end{itemize} 

\subsection{Course Engagement Expectations}

The weekly workload for this course will vary by student but on average should be about 11.5 hours per week.  The follow tables provides a rough estimate of the distribution of this time over different course components for a 15 week semester. 
\begin{center}
\begin{tabular}{|l|l|l|}
\hline
Lectures+Final &           & 3 hours/week \\ 
Homework/Interpreters & 60 hours  & 4 hours/week \\
Exam Study Time & 8 hours  & 0.5 hours/week \\ 
Quiz Study Time & 8 hours & 0.5 hours/week \\
Paper/Presentation & 22.5 hours & 1.5 hours/week \\
Reading+Unstructured Study & & 2 hours/week \\
\hline
& & 11.5 hours/week \\ 
\hline
\end{tabular}
\end{center}


\subsubsection{Calendar}

The following calendar should give you a feel for how work is distributed throughout the semester.  Assignments and events are listed in the week they are due or when the occur. \textit{This calendar is subject to change based on the circumstances of the course.}

\begin{center}
\begin{tabular}{|c|c|r|}
\hline 
Week & Dates & Assignments \\
\hline
1 & 8/25 - 8/28 &  \\
\hline 
2 & 8/31 - 9/4 &  Written Homework Due  \\
\hline
3 & 9/7 - 9/11 &   Interp. 1 \\
\hline
4 & 9/14 - 9/18 &   Quiz 1. \\
\hline
5 & 9/21 - 9/25 &  Interp. 2\\
\hline
6 & 9/28 - 10/2 & \\
\hline
7 & 10/5 - 10/9  &  Interp 3. Paper Anno. Bib. Hwk. \\
\hline 
8 & 10/12 - 10/15 & Midterm Exam.  FALL BREAK (F) \\
\hline
9 & 10/21 - 10/23 & FALL BREAK (M,Tu). \\
\hline
10 & 10/26 - 10/30 &  \\
\hline
11 & 11/2 - 11/6 & Interp 4. Quiz 2.\\
\hline
12 & 11/9 - 11/13 &  Paper Due.\\
\hline
13 & 11/16 - 11/20 & Paper Peer-Review Due. Interp 5.\\
\hline
14 & 11/23 - 11/24 &  Paper Presentations. THANKSGIVING BREAK (W-F).   \\
\hline
15 & 11/30 - 12/4 & Interp 6. Quiz 3. \\ 
\hline
16 & 12/7 - 12/9 &   Reading Day (Th). \\
\hline
Final's Week & 12/12 (6:30-9:30pm) & Final Exam. \\ 
\hline
\end{tabular}
\end{center}


\end{document}