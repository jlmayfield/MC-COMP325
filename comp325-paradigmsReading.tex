\documentclass[10pt]{article}
\usepackage{amsmath}
\usepackage{setspace}
\usepackage{hyperref}

\setlength{\textheight}{9in} \setlength{\topmargin}{-.5in}
\setlength{\textwidth}{6.5in} \setlength{\oddsidemargin}{0in}
\setlength{\evensidemargin}{0in}

\title{COMP325 - Programming Languages in Theory and Practice}
\author{  }
\date{Fall 2013}

\begin{document}
\maketitle

\section{Paradigms}

To get you thinking about Programming Languages, read Robert Floyd's Turing Lecture \textit{The Paradigms of Programming} and the first three sections of Peter Van Roy's \textit{Programming Paradigms for Dummies: What Every Programmer Should Know}.  First things first: look up the author's and find out who they are and what their credentials are.  

Now, as you read look for insight on the following questions:
\begin{enumerate}
\item \textit{What is a Programming Paradigm?}
\begin{itemize}
\item list concrete examples
\item list the authors' definitions. (compare and contrast)
\item What is their purpose? 
\item From where do they originate? 
\item To what extent is the distinction technical vs. social?
\item \textbf{Definition in your own words}
\end{itemize}
\item \textit{What's the relationship between programming languages and paradigms?}
\begin{itemize}
\item practice vs theory
\end{itemize}
\item \textit{How can programmers and program language designers invent/discover new paradigms and programming concepts?}
\end{enumerate}

\section{Language Design and Community}

Guy Steele is a big name in the Programming Languages community.  Go look up him up.  Now, back in 1998 he gave the keynote speech at OOPSLA (go look that up!) and that speech has becoming a bit of a rallying cry for a segment of the language design community.  Watch it here: \url{http://www.youtube.com/watch?v=_ahvzDzKdB0}. You've also been given a transcript of the talk. Think about Steele's comments in comparison to that of Floyd's and Van Roy's.


\section{Case Studies}

The C language and the Lisp family of languages are some of the most famous, widely known, and used languages in computing history.  The story of their early development is documented in Richie's \textit{The Development of the C Language} and McCarthy's \textit{History of Lisp}.  As always, go look up the authors.  Your goal for the reading of these papers is simple: \textit{understand the context in which the languages were built and the goals of the designers}. Once you've done that, \textit{compare and contrast what you've found with the commentary of Floyd, Van Roy, and Steele}.  

\section{The Response}

\begin{center}
\textbf{Due Friday 8/30}
\end{center}

Once you've completed the readings and viewings, write a one page response to what you've seen and read.  Using the above prompts as a guide, try to draw some kind of consistent thread between everything.  Your goal is not to answer to the questions directly but look for some deeper insight that comes from the answers you've found and the comparisons you've drawn.

\end{document}
